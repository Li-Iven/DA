\section{Тест производительности}
{\itshape Тут Вы описываете собственно тест производительности, сравнение Вашей реализации с уже существующими и т.д.}


Тест производительности представляет из себя следующее: мы сортируем начальный вектор пар \enquote{ключ - значение} двумя сортировками: нашей поразрядной сортировкой и втроеноой сортировкой std::stable\_sort.

\vspace{\baselineskip}

Для этого создаём программу для генерации тестов. 

\vspace{\baselineskip}
\begin{lstlisting}[language=Python]
import sys
import random
import string

def get_random_string():
    length = random.randint(1, 64)
    random_list = [ random.choice(string.ascii_letters) for _ in range(length) ]
    return "".join(random_list)

def main():
    n = random.randint(100, 100)
    for _ in range(n):
        a = random.randint(1, 31)
        b = random.randint(1, 12)
        c = random.randint(1, 9999)
        value = get_random_string()
        print(f"{a}.{b}.{c} {value}")
main()
\end{lstlisting}

\vspace{\baselineskip}

И проверяем время работы сортировок (с помощью библиотеки chrono - она позволяет замерить время от запуска сортировки до её завершения). Причём проверям на одинаковых данных, но разных рамеров.

\begin{alltt}
parsifal@DESKTOP-3G70RV4:~/DA/lab1/make/solution\$ python3 generator.py > 1.t
parsifal@DESKTOP-3G70RV4:~/DA/lab1/make/solution\$ g++ benchmark1.cpp
parsifal@DESKTOP-3G70RV4:~/DA/lab1/make/solution\$ ./a.out < 1.t
Count of lines is 10000
Counting sort time: 3999us
STL stable sort time: 4170us
\end{alltt}

{\setlength{\extrarowheight}{7pt}
\begin{center}
\begin{tabular}{|m{3cm}|m{6cm}|}
\hline
\rowcolor{lightgray}
Количество строк в тесте & Время работы сортировок (мкс)\\
\hline
100 &Counting sort time: 150 \newline STL stable sort time: 25\\
\hline
1000 &  Counting sort time: 283 \newline STL stable sort time: 332 \\
\hline
5000 & Counting sort time: 1774 \newline STL stable sort time: 1866 \\
\hline
10.000 & Counting sort time: 3999 \newline STL stable sort time: 4170\\
\hline
100.000 & Counting sort time: 39795 \newline STL stable sort time: 49599\\
\hline
1.000.000 & Counting sort time: 411158 \newline STL stable sort time: 705164\\
\hline
\end{tabular}
\end{center}
}

\vspace{\baselineskip}

Так как в данном случае у нас m - константа, то сложность поразрядной сортировки становится $O(d*n)$, где n - число строк в тестовом файле, а $d=3$ - число разрядов.
Можно увидеть, что при количестве текстов (n) большем или равном 1000 поразрядная сортировка опережает встроенную. При значениях же меньших 1000 встроенная сортировка опережает поразрядную, т.к. $O(3*n)<O(log(n)*n=O(2*n)$

\pagebreak