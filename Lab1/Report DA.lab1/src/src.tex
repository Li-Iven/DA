\section{Описание}
Требуется написать реализацию алгоритма поразрядной сортировки.

\vspace{\baselineskip}

Как сказано в \cite{Kormen}: \enquote{в алгоритме поразрядной сортировки мы предполагаем, что каждый ключ сортировки можно рассматривать как d-значное число, каждая цифра которого находится в диапазоне от О до $m -1$. Мы поочередно используем устойчивую сортировку для каждой цифры справа налево. Если в качестве устойчивой применяется сортировка подсчетом, то время сортировки по одной цифре составляет $O( m + n )$, а время сортировки по всем d цифрам - $O( d ( m + n))$. Если т является константой ..., то время работы поразрядной сортировки становится равным $O( d*n )$. Если d также представляет собой константу ..., то время работы поразрядной сортировки превращается в просто O(n).}

\vspace{\baselineskip}

Я буду следовать этому алгоритму. В качестве устойчивой сортировки возьму сортировку подсчётом. А в качестве разрядов - дни, месяцы и годы в датах.

\vspace{\baselineskip}

Единственное отличие моей сортировки от описанной выше - это то, что я буду идти по разрядам ключа (даты) не справа на лево, а наоборот. Т.к. в случае чисел правый разряд имеет \enquote{наименьшее} значение, и в случае дат, именно день также имеет  \enquote{наименьшее} значение.

\pagebreak

\section{Исходный код}

На каждой непустой строке файла располагается пара \enquote{ключ - значение}, поэтому создаем структуру TPair, где хранятся ключ и значение.

\vspace{\baselineskip}

Чтобы сортировка работала быстро, нам нужно иметь доступ к произвольному элементу за О(1). Для этого реализуем шаблон класса TVector, который по функционалу похож на std::vector. В нём мы будем хранить наши пары \enquote{ключ - значение}.

Рассмотрим алгоритм сортировки: 
\begin{enumerate}
    \item На вход поступает набор элементов (по ссылке) и максимальное значение среди этого набора (по константной ссылке).
    \item Разбиваем ключ на три разряда - день, месяц, год.
    \item Для каждого из трёх разрядов осуществляем сортировку подсчётом. Начинаем с наименее значимого разряда - с дней.
    \begin{itemize}
        \item Создаём временный вектор чисел tmp (его размер на 1 больше максимального элемента). Инициализируем его нулями. 
        \item Далее считаем сколько раз каждое число встретилось в начальном векторе.
        \item Затем к каждому элементу вектора tmp прибавляем его предыдущий элемент.
        \item Создаём результирующий вектор. Заполняем его.
        \item Меняем местами значения начального вектора и результирующего.
    \end{itemize}
    \item В результате работы сортировки мы получаем изменённый вектор.
\end{enumerate}

Сам код:

\vspace{\baselineskip}

\begin{lstlisting}[language=C]
#include <iostream>
#include <cstdio>
#include <string.h>
template <typename T>
class TVector{
public:
	TVector() = default;
	~TVector() {
		delete[] arr;
	}
	TVector(const size_t len){
		arr = new T[len];
		size = len;
		capecity = len;
	}
	TVector(const size_t len, const T value){
		arr = new T[len];
		size = len;
		capecity = len;
		for (size_t i = 0; i < size; i++) {
			arr[i] = value;
		}
	}
	void PushBack(const T& elem){
		if (size == capecity) {
			capecity++;
			capecity = capecity*2;
			size_t new_size = capecity;
			T* tmp = new T[new_size];
			std::copy(begin(), end(), tmp);
			delete[] arr;
			arr = tmp;
		}
		arr[size] = elem;
		size++;
	}
	void swap(TVector& b){
		T* c = arr;
		arr = b.arr;
		b.arr = c;
	}
	T& operator[] (const int i){
		return arr[i];
	}
	TVector& operator=(TVector& other) {
		if (&other == this) {
			return *this;
		}
		if (other.size <= capecity) {
			std::copy(other.begin(), other.end(), begin());
			size = other.size;
		}
		else {
			TVector<T> tmp(other);
			std::swap(tmp.arr, arr);
			std::swap(tmp.size, size);
			std::swap(tmp.capecity, capecity);
		}
		return *this;
	}
	size_t Size() const {
		return size;
	}
	T* begin(){
		return arr;
	}
	T* end(){
		return arr + size;
	}

private:
	size_t size = 0;
	size_t capecity = 0;
	T* arr = nullptr;
};

struct TPair{
    int znacheniye[3] = {0};
    char stroka[11] = {'\0'};
    char val[65];
};

int RadixSort(TVector<TPair> &elems, const int (&max)[3]) {
	size_t N = elems.Size();
    if (N == 0) {
        return 0;
	}
	for (int i = 0; i < 3; ++i) {
		TVector<int> tmp((max[i] + 1),0);
		for (size_t j = 0; j < N; ++j) {
			tmp[elems[j].znacheniye[i]]++;
		}

		for (size_t j = 1; j < tmp.Size(); ++j) {
			tmp[j] += tmp[j-1];
		}

		TVector<TPair> res(N);
		for (int j = N-1; j > (-1); j--) {
			res[tmp[elems[j].znacheniye[i]] - 1] = elems[j];
			--tmp[elems[j].znacheniye[i]];
		}
		
		elems.swap(res);
	}
	return 0;
}


int main(){
	std::ios::sync_with_stdio(false);
    std::cin.tie(nullptr);
    std::cout.tie(nullptr);
	int max[3] = {0};
	
	TVector<TPair> elems;
	TPair x;

	while (scanf("%s %s", x.stroka, x.val)>0) {
		sscanf(x.stroka, "%d.%d.%d", &x.znacheniye[0], &x.znacheniye[1], &x.znacheniye[2]);
        elems.PushBack(x);
		if(x.znacheniye[0] > max[0]) max[0] = x.znacheniye[0];
		if(x.znacheniye[1] > max[1]) max[1] = x.znacheniye[1];
		if(x.znacheniye[2] > max[2]) max[2] = x.znacheniye[2];
    }
	
	RadixSort(elems, max);

	for (size_t i = 0; i < elems.Size(); ++i) {
        std::cout << elems[i].stroka << " " << elems[i].val << "\n";
    }
	return 0;
}
\end{lstlisting}

\pagebreak

\section{Консоль}
\begin{alltt}
parsifal@DESKTOP-3G70RV4:~/DA/lab1/make/solution\$ g++ main.cpp
parsifal@DESKTOP-3G70RV4:~/DA/lab1/make/solution\$ cat test.t
1.1.1 n399tann9antt3tn93aat3naatt
01.02.2008  n399n9nann93na9t3an9antt3tn93aat3naat
1.1.1 n399tann9nnt3ttnaaan9nann93nt3tn93aat3naa
01.02.2008  n399tann9nann93na9t3a3t9999na3a3na
23.12.9999 ubintsthrstjry
01.1.1 hhhhhhhhhhhhhhh
parsifal@DESKTOP-3G70RV4:~/DA/lab1/make/solution\$ ./a.out < test.t
1.1.1 n399tann9antt3tn93aat3naatt
1.1.1 n399tann9nnt3ttnaaan9nann93nt3tn93aat3naa
01.1.1 hhhhhhhhhhhhhhh
01.02.2008 n399n9nann93na9t3an9antt3tn93aat3naat
01.02.2008 n399tann9nann93na9t3a3t9999na3a3na
23.12.9999 ubintsthrstjry
\end{alltt}
\pagebreak
