\section{Описание}
Требуется написать реализацию Z-функции для поиска подстроки в строке (образца в тексте).

\vspace{\baselineskip}

Справка вики\cite{wikipedia_z}: {\bfseries Z-фуннкция от строки S} — массив $Z_{1},\dots,Z_{n}$ такой что $Z_{i}$ равен длине наибольшего общего префикса начинающегося с позиции $i$ суффикса строки $S$ и самой строки $S$.

\vspace{\baselineskip}

Рассмотрим алгоритм вычисления Z-функции за линейное время:

\begin{enumerate}
    \item Назовём отрезком совпадения подстроку, совпадающую с префиксом $S$. Будем поддерживать координаты $l$ и $r$ самого правого отрезка совпадения. Пусть $i$ - текущий индекс, для которого мы хотим вычислить $Z_i(S)$.
    \item Если $i <= r$ - попали в отрезок совпадения, так как строки совпадают, то и Z-блоки для них по отдельности совпадают $\Rightarrow Z_i(S) = z_{i - l}$. Так как $i + Z_i(S)$ может быть за пределами отрезка совпадения, то нужно ограничить значение величиной $r - i + 1$.
    \item $i > r$ - тривиальный алгоритм (прсто прикладываем паттер к тексту , каждый раз сдвигая его на один символ). 
    \item В конце обновляем отрезок совпадения, если $i + Z_i(S) > r$ (тривиальный алгортим вышел за отрезок совпадения): $l = i, r = i + Z_i(S) - 1$.
\end{enumerate}

\vspace{\baselineskip}
Теперь рассмотрим алгоритм поиска подстроки в строке с помощью Z-фукнкции: 

\vspace{\baselineskip}

\begin{enumerate}
    \item Во избежании путаницы назовём одну строку {\bfseries текстом $t$}, а другую - {\bfseries образцом $p$}. Таким образом, задача заключается в том, чтобы найти все вхождения образца $p$ в текст $t$.
    \item Для решения этой задачи образуем строку $s = p + \# + t$, т.е. к образцу припишем текст через символ-разделитель (который не всечается нигде в самих строках). Так как у нас слова могут состоять из букв латинского алфавита, а в теексте могу встречаться цифры, то в кчестве символа-разделителя я буду использовать знак $\$$.
    \item Поссчитаем для полученной строки Z-функцию. Тогда для любого $i$ в отрезке $[0; length(t) - 1]$ по соответствующему значению $z[i + length(p) + 1]$ можно понять, вхлдит ли образец $p$ в текст $t$, начиная с позиции $i$: если это значение Z-функции равно $length(p)$, то да, входит, иначе - нет.
\end{enumerate}

\vspace{\baselineskip}
Таким образом, асимптотика решения получилась $O(length(t) + length(p))$. Потребление памяти имеет ту же асимптотику.

\pagebreak

\section{Исходный код}

В начале создаём структуру случайного слова {\bfseries TWord}, в которой будут лежать строка, храниящая собственно само слово, и два числа - индекс строки, в которой находится этого слово, и его порядковый номер в ней.

\vspace{\baselineskip}

Далее реализуем саму Z-функцию, а так же вспомогательную функцию перевода букв слов в сторочные символы (для реализации регистронезависимости нашей программы).

\vspace{\baselineskip}
Потом в main посимвольно считываем сначала образец, а потом текст (посимвольное считывание нужно для правильной обработки знаков пробела, переноса строк, конца файла, а так же самих букв в словах). Парралельно с каждым знаком ' ' мы увеличиваем наш счётчик слов, а при переносе строки обнуляем его и увеличиваем счётчик строк. После считывания образца не забываем добавить к результирующей строке $s$ символ $\$$.

\vspace{\baselineskip}

Сам код:

\vspace{\baselineskip}

\begin{lstlisting}[language=C++]
#include <iostream>
#include <cstring>
#include <vector>
const unsigned int WORD_SIZE = 17;

struct TWord {
    std::string Word = "";
    unsigned int WordIndex;
    unsigned int LineIndex;

    TWord() = default;
    TWord(std::string &s, unsigned int wi, unsigned int li) {
        Word = s;
        WordIndex = wi;
        LineIndex = li;
    };
    ~TWord() = default;

    const bool operator==(const TWord &rhs) const {
        for (unsigned int i = 0; i < WORD_SIZE; ++i) 
            if (Word[i] != rhs.Word[i])
                return false;
            else return true;
    }
};

std::vector<unsigned int> ZFunction(const std::vector<TWord> &text) {
    unsigned int TextSize = text.size();
    unsigned int left = 0;
    unsigned int right = 0;
    std::vector<unsigned int> result(TextSize);
    for (unsigned int i = 1; i < TextSize; ++i) {
        if (i <= right) { 
            result[i] = std::min(result[i-left], right - i);
        }
        while ((result[i] + i < TextSize) && (text[result[i]].Word == text[result[i] + i].Word)) {
            ++result[i];
        }
        if (i + result[i] > right) {
            left = i;
            right = result[i] + i;
        }
    }
    return result;
}

void ToLower(char* str) {
    if (*str>='A' and *str<='Z')
        *str -= 'A'-'a';
}

int main() {

    char c;
    std::string word = "";
    unsigned int PatternSize = 0;
    unsigned int WordNumber = 0;
    unsigned int LineNumber = 1;
    std::vector<TWord> text;

    while ((c = getchar()) != EOF) {
        if (c == '\n') {
            if (word != "") {
                ++WordNumber;
                text.push_back(TWord(word,WordNumber,0));
                ++PatternSize;
                word = "";
            }
            break;
        }
        else if (c == ' ') {
            if (word != "") {
                ++WordNumber;
                text.push_back(TWord(word,WordNumber,0));
                ++PatternSize;
                word = "";
            }
        }
        else {
            ToLower(&c);
            word += c;
        }
    }
    
    if (PatternSize == 0) return 0;

    std::string x = std::string("$");
    text.push_back(TWord(x,0,0));

    WordNumber = 0;

    while ((c = getchar()) != EOF) {
        if (c == '\n') {
            ++WordNumber;
            if (word != "") {
                text.push_back(TWord(word,WordNumber,LineNumber));
                word = "";
            }
            ++LineNumber;
            WordNumber = 0;
        }
        else if (c == ' ') {
            if (word != "") {
                ++WordNumber;
                text.push_back(TWord(word,WordNumber,LineNumber));
                word = "";
            }
        }
        else {
            ToLower(&c);
            word +=c;
        }
    }

    std::vector<unsigned int> Result = ZFunction(text);

    for (int i = 0; i < text.size(); i++) {
        if (Result[i] == PatternSize) std::cout << text[i].LineIndex << ", " << text[i].WordIndex << "\n";
    }
    return 0;
}
\end{lstlisting}

\vspace{\baselineskip}

\pagebreak

\section{Консоль}
\begin{alltt}
parsifal@DESKTOP-3G70RV4:~/DA/Lab4\$ python3 generator.py
Pattern =  ['dog', 'cat', 'dog', 'dog', 'bird']
parsifal@DESKTOP-3G70RV4:~/DA/Lab4\$ cat test01.t
dog cat dog dog bird
doctor parzival Yatty AttackHelicopter parzival cat
dog dog
cat
dog dog bird
doctor AttackHelicopter parzival doctor Yatty parzival dog cat
dog dog bird 
parsifal@DESKTOP-3G70RV4:~/DA/Lab4\$ g++ main.cpp
parsifal@DESKTOP-3G70RV4:~/DA/Lab4\$ ./a.out  < test01.t
2, 2
5, 7
\end{alltt}
\pagebreak
