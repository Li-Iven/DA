\CWHeader{Лабораторная работа \textnumero 4}

\CWProblem{ 

\vspace{\baselineskip}

{\bfseries Вариант \textnumero4-1}

Необходимо реализовать один из стандартных алгоритмов поиска образцов для указанного алфавита.

Вариант алгоритма: Поиск одного образца основанный на построении Z-блоков.

Вариант алфавита: Слова не более 16 знаков латинского алфавита (регистронезависимые).

Запрещается реализовывать алгоритмы на алфавитах меньшей размерности, чем указано в задании.

Программа должна обрабатывать строки входного файла до его окончания. Каждая строка может иметь следующий формат:

\vspace{\baselineskip}

{\bfseries Формат входных данных} 

Искомый образец задаётся на первой строке входного файла.

Затем следует текст, состоящий из слов или чисел, в котором нужно найти заданные образцы.

Никаких ограничений на длину строк, равно как и на количество слов или чисел в них, не накладывается.

\vspace{\baselineskip}

{\bfseries Формат результата} 

В выходной файл нужно вывести информацию о всех вхождениях искомых образцов в обрабатываемый текст: по одному вхождению на строку.

Следует вывести два числа через запятую: номер строки и номер слова в строке, с которого начинается найденный образец.

Нумерация начинается с единицы. Номер строки в тексте должен отсчитываться от его реального начала (то есть, без учёта строк, занятых образцами).

Порядок следования вхождений образцов несущественен.

}
\pagebreak