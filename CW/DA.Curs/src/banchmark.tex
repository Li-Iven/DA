\section{Коэффициент сжатия, время работы}

Для тестирования программы напишем генератор текстов из малых латинских букв, где мы можем задавать длинну повторений и их количество.

\vspace{\baselineskip}
\begin{lstlisting}[language=Python]
import random
import string


def generate_random_string(length):
    letters = string.ascii_lowercase
    rand_string = ''.join(random.choice(letters) for i in range(length))
    rand_string = rand_string*100
    print(rand_string)

print("compress")
generate_random_string(50000)
\end{lstlisting}

\vspace{\baselineskip}

И проверяем время работы архиватора на текстах различной длинны 
(будем изменять параметр функции generate\_random\_string()
rand\_string = rand\_string*200).

{\setlength{\extrarowheight}{7pt}
\begin{center}
\begin{tabular}{|m{4cm}|m{4cm}|}
\hline
\rowcolor{lightgray}
Количество символов & Время работы архиватора (мс)\\
\hline
1000 & 12\\
\hline
10.000 & 14 \\
\hline
100.000 & 34  \\
\hline
1.000.000 & 417 \\
\hline
10.000.000 & 5023 \\
\hline
100.000.000 & 69761\\
\hline
\end{tabular}
\end{center}
}

\pagebreak
Коэффициент сжатия (при rand\_string = rand\_string*100):

{\setlength{\extrarowheight}{7pt}
\begin{center}
\begin{tabular}{|m{3cm}|m{3cm}|m{3cm}|m{3cm}|}
\hline
\rowcolor{lightgray}
Количество символов & Размер начального файла(Кб) & Размер конечного файла(Кб) & Коэффициент сжатия\\
\hline
1000 & 1 & 1 & 1\\
\hline
10.000 & 10 & 6 & 1,67\\
\hline
100.000 & 98 & 64 & 1,53\\
\hline
1.000.000 & 977 & 723 & 1,35\\
\hline
10.000.000 & 9766 & 7962 & 1,23\\
\hline
100.000.000 & 97657 & 86387 & 1,13\\
\hline
\end{tabular}
\end{center}
}

\pagebreak