\section{Теория}
Требуется написать реализацию алгоритма сжатия LZW.

\vspace{\baselineskip}
Алгориитм Лемпеля — Зива — Уэлча (Lempel-Ziv-Welch, LZW) — это универсальный алгоритм сжатия данных без потерь, созданный Авраамом Лемпелем (англ. Abraham Lempel), Яаковом Зивом (англ. Jacob Ziv) и Терри Велчем (англ. Terry Welch). Он был опубликован Велчем в 1984 году в качестве улучшенной реализации алгоритма LZ78, опубликованного Лемпелем и Зивом в 1978 году\cite{wikipedia}. (Включение явного символа в вывод после каждой фразы часто является расточительным. В LZW этого избегают с помощью инициализации списка фраз, включающего все символы исходного алфавита.)


\vspace{\baselineskip}
Данный алгоритм при кодировании (сжатии) сообщения динамически создаёт словарь фраз: определённым последовательностям символов (фразам) ставятся в соответствие группы битов (коды). Словарь инициализируется всеми 1-символьными фразами.

\vspace{\baselineskip}
Формально данный алгоритм можно описать следующей последовательностью шагов:

\begin{enumerate}
    \item Инициализация словаря всеми возможными односимвольными фразами. Инициализация входной фразы W первым символом сообщения.
    \item Если КОНЕЦ\_СООБЩЕНИЯ, то выдать код для W и завершить алгоритм.
    \item Считать очередной символ K из кодируемого сообщения.
    \item Если фраза WK уже есть в словаре, то присвоить входной фразе W значение WK и перейти к Шагу 2.
    \item Иначе выдать код W, добавить WK в словарь, присвоить входной фразе W значение K и перейти к Шагу 2.
\end{enumerate}

\vspace{\baselineskip}
Алгоритму декодирования на входе требуется только закодированный текст: соответствующий словарь фраз легко воссоздаётся посредством имитации работы алгоритма кодирования

\begin{enumerate}
    \item Инициализация словаря всеми возможными односимвольными фразами. Инициализация входного кода С первым числом сообщения.
    \item Если КОНЕЦ\_СООБЩЕНИЯ, то выдать фразу для С и завершить алгоритм.
    \item Выдать фразу W, соответствующую С.
    \item Считать очередное число C2 из сообщения.
    \item Если фраза, образованная слиянием фраз кодов С и C2 = WK, уже есть в словаре, то присвоить входной фразе W значение WK и перейти к Шагу 2.
    \item Иначе добавить WK в словарь, присвоить входной фразе W значение K и перейти к Шагу 2.
    \item Если же мы встречаем код, не принадлежащий текущему, восстановленному нами алфавиту, то надо сконкатенировать предудущую строку и её первый символ.
\end{enumerate}

\pagebreak

\section{Исходный код}

В начале кодирования мы создаём словарь с ключом-фразой и значением-числом, и заполняем его начальными символами. В начале декодирования создаём тот же словарь, но уже с ключом-числом - так как мы будем уже искать фразу по коду, а не наоборот.

\vspace{\baselineskip}
Алгоритмы кодирования и декодирования были подробно описаны в разделе Теория.

\vspace{\baselineskip}
Листинг кода:

\vspace{\baselineskip}

\begin{lstlisting}[language=C]
#include <iostream>
#include <map>
#include <unordered_map>

using namespace std;

void Encoding() {

   unordered_map<string, int> Dictionary;
    for (int i = 0; i < 26; ++i) {
        string s;
        s.push_back('a' + i);
        Dictionary.insert(pair<string, int>(s, i));
        Dictionary[s] = i;
    }
    Dictionary.insert(pair<string, int>("EOF", 26));

    char c;
    string word;
    while((c = getchar())) {
        string tmp = word;
        word += c;
        if(c == EOF) {
            word = tmp;
            cout << Dictionary[tmp] << "\n";
            break;
        }
        if((Dictionary.count(word) == 0) && (c != '\n')) {
            cout << Dictionary[tmp] << " ";
            int size = Dictionary.size();
            Dictionary.insert(pair<string, int>(word, size));
            word = c;
        }
    }
}

void Decoding() {

    unordered_map<int, string> CodeDictionary;
    for (int i = 0; i < 26; ++i) {
        string s;
        s.push_back('a' + i);
        CodeDictionary.insert(pair<int, string>(i, s));
        CodeDictionary[i] = s;
    }
    CodeDictionary.insert(pair<int, string>(26, "EOF"));

    int code;
    string pattern;
    while(cin >> code) {
        string tmp = pattern;
        if(CodeDictionary.count(code) != 0) {
            pattern = CodeDictionary[code];

            if(pattern == "EOF") {
                cout << "\n";
                break;
            }

            cout  << pattern;
            if(tmp != "") {
                pattern = tmp + pattern.front();
                int size = CodeDictionary.size();
                CodeDictionary.insert(pair<int, string>(size, pattern));
                pattern = CodeDictionary[code];
            }
        }
        else {
            pattern = tmp + tmp.front();
            int size = CodeDictionary.size();
            CodeDictionary.insert(pair<int, string>(size, pattern));
            cout  << pattern;
        }
    }
}

int main() {

    string command;
    getline(cin, command);
    if(command == "compress") Encoding();
    else if(command == "decompress") Decoding();
    else cout << "Incorrect command!";

    return 0;
}
\end{lstlisting}

\pagebreak

\section{Консоль}
\begin{alltt}
parsifal@DESKTOP-3G70RV4:~/DA/Curs\$ make
g++ -std=c++17 -O3 -Wextra -Wall -pedantic main.cpp -o Curs
parsifal@DESKTOP-3G70RV4:~/DA/Curs\$ cat test.txt 
compress
acagaatagaga
parsifal@DESKTOP-3G70RV4:~/DA/Curs\$ ./Curs < test.txt
0 2 0 6 0 0 19 29 34
parsifal@DESKTOP-3G70RV4:~/DA/Curs\$ cat test.txt 
decompress
0 2 0 6 0 0 19 29 34
parsifal@DESKTOP-3G70RV4:~/DA/Curs\$ ./Curs < test.txt
acagaatagaga
\end{alltt}
\pagebreak
