\section{Тест производительности}

Тест производительности представляет из себя следующее: мы обрабатываем начальный набор данных и сохраняем пары \enquote{ключ - значение} двумя способами: нашим АВЛ-деревом и c помощью std::map.

\vspace{\baselineskip}

Для этого создаём программу для генерации тестов. 

\vspace{\baselineskip}
\begin{lstlisting}[language=Python]
import sys
import random
import string

def get_random_key():
    return random.choice(string.ascii_letters)

actions = ["+", "-", ""]

plus = 0
minus = 0
fin = 0

keys = dict()
test_file_name = "test{:02d}".format(1)
with open("{0}.t".format(test_file_name), 'w') as output_file, \
     open("{0}.a".format(test_file_name), "w") as answer_file:

    for _ in range(10):
        action = random.choice(actions)
        if action == "+":
            key = get_random_key()
            value = random.randint(1, 2 ** 64 - 1)
            output_file.write("+ {0} {1}\n".format(key, value))
            key = key.lower()
            answer = "Exists"
            if key not in keys:
                answer = "OK"
                keys[key] = value
            answer_file.write("{0}\n".format(answer))
            plus = plus + 1

        if action == "-":
            key = get_random_key()
            value = random.randint(1, 2 ** 64 - 1)
            output_file.write("- {0}\n".format(key))
            key = key.lower()
            answer = "NoSuchWord"
            if key in keys:
                answer = "OK"
                del keys[key]
            answer_file.write("{0}\n".format(answer))
            minus = minus + 1

        elif action == "":
            search_exist_element = random.choice([True, False])
            key = random.choice([key for key in keys.keys()]) if search_exist_element and len(keys.keys()) > 0 else get_random_key()
            output_file.write("{0}\n".format(key))
            key = key.lower()
            if key in keys:
                answer = "OK: {0}".format(keys[key])
            else:
                answer = "NoSuchWord"
            answer_file.write("{0}\n".format(answer))
            fin = fin + 1
    print(plus, '+, ', minus, '-, ', fin, '?')
\end{lstlisting}

\vspace{\baselineskip}

Так же создаём бенчмарк, для измерения времени работы структур (с помощью библиотеки chrono - она позволяет замерить время от запуска сортировки до её завершения). Причём проверям на одинаковых данных, но разных рамеров.

\vspace{\baselineskip}
Здесь в std::map мы храним std::string вместо TKey*, так как во втором случае она будет хранить указатели, а значит сохранит даже указатели на одинаковые ключи, т.е. будет рабоать неправильно.


\vspace{\baselineskip}
\begin{lstlisting}[language=C]
#include <iostream>
#include <map>
#include <chrono>
#include <string>
#include <cstring>
 
#include "AVL-tree3.hpp"
 
using duration_t = std::chrono::microseconds;
 
void DoLower(TKey* key) {
    int lengh = strlen(key);
    for (int i = 0; i < lengh; i++) {
        key[i] = tolower(key[i]);
    }
}
 
int main() {
    std::ios::sync_with_stdio(false);
    std::cin.tie(nullptr);
    std::cout.tie(nullptr);
 
    std::map<std::string, TValue> RB_tree;
    TAVL tree;
    TKey* key = new char[SIZE_KEY + 1];
    TValue value;
 
    std::chrono::time_point<std::chrono::system_clock> start, end;
    int64_t RB_ts = 0, AVL_ts = 0;
 
    while (std::cin >> key) {
        if (key[0] == '+') {
            std::cin >> key >> value;
            DoLower(key);
 
            start = std::chrono::system_clock::now();
            tree.Root = tree.Insert(tree.Root, key, value);
            end = std::chrono::system_clock::now();
            AVL_ts += std::chrono::duration_cast<duration_t>( end - start ).count();

            std::string key_stl = std::string(key);

            start = std::chrono::system_clock::now();
            RB_tree.insert({key_stl, value});
            end = std::chrono::system_clock::now();
            RB_ts += std::chrono::duration_cast<duration_t>( end - start ).count();
        }
        else if (key[0] == '-') {
            std::cin >> key;
            DoLower(key);
 
            start = std::chrono::system_clock::now();
            tree.Root = tree.Remove(tree.Root, key);
            end = std::chrono::system_clock::now();
            AVL_ts += std::chrono::duration_cast<duration_t>( end - start ).count();

            std::string key_stl = std::string(key);
 
            start = std::chrono::system_clock::now();
            RB_tree.erase(key_stl);
            end = std::chrono::system_clock::now();
            RB_ts += std::chrono::duration_cast<duration_t>( end - start ).count();
        }
        else {
            DoLower(key);
 
            start = std::chrono::system_clock::now();
            tree.PrintFind(key);
            end = std::chrono::system_clock::now();
            AVL_ts += std::chrono::duration_cast<duration_t>( end - start ).count();

            std::string key_stl = std::string(key);
 
            start = std::chrono::system_clock::now();
            RB_tree.find(key_stl);
            end = std::chrono::system_clock::now();
            RB_ts += std::chrono::duration_cast<duration_t>( end - start ).count();
        }
        key = new char[SIZE_KEY + 1];
    }
 
    std::cout << "STL std::map time: " << RB_ts << "\nAVL-tree time: " << AVL_ts << std::endl;
    return 0;
}
\end{lstlisting}

\begin{alltt}
parsifal@DESKTOP-3G70RV4:~/DA/Lab2\$ python3 generator_tests.py
33344 +,  33241 -,  33415 ?
parsifal@DESKTOP-3G70RV4:~/DA/Lab2\$ g++ benchmark_new.cpp
parsifal@DESKTOP-3G70RV4:~/DA/Lab2\$ ./a.out < test01.t
В main массив на куче --- В map хранится std::string
STL std::map time: 53131
AVL-tree time: 29447
\end{alltt}

\vspace{\baselineskip}

{\setlength{\extrarowheight}{7pt}
\begin{center}
\begin{tabular}{|m{3cm}|m{6cm}|}
\hline
\rowcolor{lightgray}
Количество строк в тесте & Время работы структур данных (мкс)\\
\hline
100 &STL std::map time: 74 \newline AVL-tree time: 34\\
\hline
1000 &  STL std::map time:: 853 \newline AVL-tree time: 553 \\
\hline
10.000 & STL std::map time: 6875 \newline AVL-tree time: 4289\\
\hline
100.000 & STL std::map time: 48198 \newline AVL-tree time: 24185\\
\hline
1.000.000 & STL std::map time: 489087 \newline AVL-tree time: 222993\\
\hline
\end{tabular}
\end{center}
}

\vspace{\baselineskip}

Сложности вставки, удаления и поиска в std::map и в АВЛ-деревьях одинаковые - $O(log(n))$ (так как std::map основана на красно-чёрных деревьях). Однако у меня вышло, что время выполнения std::map почти в 2 раза выше, чем выполнение АВЛ-дерева.

\vspace{\baselineskip}

Я думаю это может быть из-за того, что, $O(log(n))$ - это общая оценка, а не точная. Точная же оценка сверху у этих алгоритмов может быть $O(log(n) + 9999)$ или $O(log(n*9999))$ и т.д. И в нашем случае эти 9999 могли сыграть свою роль, замедлив std::map.

\pagebreak