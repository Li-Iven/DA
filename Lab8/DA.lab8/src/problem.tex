\CWHeader{Лабораторная работа \textnumero 8}

\CWProblem{ Топологическая сортировка

\vspace{\baselineskip}

{\bfseries Вариант \textnumero 6}

Разработать жадный алгоритм решения задачи, определяемой своим вариантом. Доказать его корректность, оценить скорость и объём затрачиваемой оперативной памяти.

Реализовать программу на языке С или С++, соответсвующую построенному алгоритму. Формат входных и выходных данных описан в варианте задания.

Дан набор из {\bfseries N} объектов. Каждому из объектов присвоен свой номер: {\bfseries 1, 2, 3} и т.д. Кроме того, заданы {\bfseries N} условий с ограничиениями на расположение вида «{\bfseries A} должен находиться перед {\bfseries B}». Необходимо найти такой минимальный набор правил, что все остальные ограничения будут выполняться.

\vspace{\baselineskip}

{\bfseries Формат входных данных} 

На первой строке два числа, {\bfseries N} и {\bfseries M}, за которыми следует {\bfseries M} строк с ограничениями вида «{\bfseries A < B}» ($1 \leq A, B \leq N$ ) определяющими относительную последовательность объектов с номерами {\bfseries A} и {\bfseries B}.

\vspace{\baselineskip}

{\bfseries Формат результата} 

{\bfseries \--1}, если расположить объекты в соответствии с требованиями невозможно, искомый набор правил в противном случае.

}

\pagebreak