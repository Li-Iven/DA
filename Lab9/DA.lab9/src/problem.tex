\CWHeader{Лабораторная работа \textnumero 9}

\CWProblem{ Поиск кратчайшего пути между парой вершин алгоритмом Дейкстры

\vspace{\baselineskip}

{\bfseries Вариант \textnumero 4}

Разработать программу на языке C или C++, реализующую указанный алгоритм согласно заданию:

Задан взвешенный неориентированный граф, состоящий из {\bfseries n} вершин и {\bfseries m} ребер. Вершины пронумерованы целыми числами {\bfseries от 1 до n}. Необходимо найти длину кратчайшего пути из вершины с номером {\bfseries start} в вершину с номером {\bfseries finish} при помощи алгоритма Дейкстры. Длина пути равна сумме весов ребер на этом пути. Граф не содержит петель и кратных ребер.

\vspace{\baselineskip}

{\bfseries Формат входных данных} 

В первой строке заданы {\bfseries $1 \leq n \leq 10^5, 1 \leq m \leq 10^5, 1 \leq start \leq n$ и $1 \leq finish \leq n$}. В следующих {\bfseries m} строках записаны ребра. Каждая строка содержит три числа – номера вершин, соединенных ребром, и вес данного ребра. Вес ребра – целое число {\bfseries от 0 до $10^9$}.

\vspace{\baselineskip}

{\bfseries Формат результата} 

Необходимо вывести одно число – длину кратчайшего пути между указанными вершинами. Если пути между указанными вершинами не существует, следует вывести строку {\bfseries "No solution"\,} (без кавычек).

}

\pagebreak