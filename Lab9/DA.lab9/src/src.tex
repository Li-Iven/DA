\section{Описание}
Требуется написать реализацию алгоритма Дейкстры поиска кратчайшего пути между парой вершин в неориентированном взвешанном графе.

\vspace{\baselineskip}

Справка вики\cite{wiki}: {\bfseries Алгоритм Дейкстры} — находит кратчайшие пути от одной из вершин графа до всех остальных. Алгоритм работает только для графов без рёбер отрицательного веса.

\vspace{\baselineskip}

{\bfseries Неформальное объяснение:}

Каждой вершине из $V$ сопоставим метку — минимальное известное расстояние от этой вершины до $a$.

Алгоритм работает пошагово — на каждом шаге он «посещает» одну вершину и пытается уменьшать метки.

Работа алгоритма завершается, когда все вершины посещены.

\vspace{\baselineskip}
{\bfseries Инициализация.}

\begin{itemize}
    \item Метка самой вершины $a$ полагается равной 0, метки остальных вершин — бесконечности.
    \item Это отражает то, что расстояния от $a$ до других вершин пока неизвестны.
    \item Все вершины графа помечаются как непосещённые.
\end{itemize}

\vspace{\baselineskip}
{\bfseries Шаг алгоритма.}

\begin{itemize}
    \item Если все вершины посещены, алгоритм завершается.
    \item В противном случае, из ещё не посещённых вершин выбирается вершина $u$, имеющая минимальную метку.
    \item Мы рассматриваем всевозможные маршруты, в которых $u$ является предпоследним пунктом. Вершины, в которые ведут рёбра из $u$, назовём \emph{соседями} этой вершины. Для каждого соседа вершины $u$, кроме отмеченных как посещённые, рассмотрим новую длину пути, равную сумме значений текущей метки $u$ и длины ребра, соединяющего $u$ с этим соседом.
    \item Если полученное значение длины меньше значения метки соседа, заменим значение метки полученным значением длины. Рассмотрев всех соседей, пометим вершину $u$ как посещённую и повторим шаг алгоритма.
\end{itemize}

\pagebreak

\section{Исходный код}

В начале создадим маленькую стрктурку ребра - она будет состоять из номера вершины, в котрое оно ведёт от текущей, и из стоимости ребра. Далее создаём граф - вектор векторов рёбер.

Для учёта незанятых вершин используем сет, так как в нём удобно находить минимальный элемент - по указателю на первый элемент.

\vspace{\baselineskip}

Сам код:

\vspace{\baselineskip}

\begin{lstlisting}[language=C++]
#include <iostream>
#include <vector>
#include <set>

using namespace std;

const long long INF = 1e13 + 7;

struct TEdge {
    int to, cost;
};

int main() {

    ios::sync_with_stdio(false);
    cin.tie(nullptr);
    cout.tie(nullptr);

    int n, m, s, f;
    cin >> n >> m >> s >> f;
    s--;
    f--;

    int a, b, c;
    vector<vector<TEdge>> G(n);
    for (int i = 0; i < m; ++i) {
        cin >> a >> b >> c;
        a--;
        b--;
        G[a].push_back({b, c});
        G[b].push_back({a, c});
    }

    long long dis = 0;
    vector<long long> d(n, INF);
    d[s] = 0;

    set<pair<long long, int>> st; // 1 -- dist, 2 -- num
    st.insert({0, s});

    while (!st.empty()) {

        pair<long long,int> p = *st.begin();
        long long dist = p.first;
        int current = p.second;
        if (current == f) {
            dis = dist;
            break;
        }
        st.erase(st.begin());
        
        for (TEdge to_cost: G[current]) {
            int to = to_cost.to;
            long long cost = to_cost.cost;
            if (d[to] > d[current] + cost) {
                st.erase({d[to], to});
                d[to] = d[current] + cost;
                st.insert({d[to], to});
            }
        }  
    }

    if (!st.empty()) {
        cout << dis << "\n";
    } else {
        cout << "No solution" << "\n";
    }

    return 0;
}
\end{lstlisting}

\vspace{\baselineskip}

\pagebreak

\section{Консоль}
\begin{alltt}
parsifal@DESKTOP-3G70RV4:~/DA/Lab9\$ make
g++ -std=c++17 -O3 -Wextra -Wall -pedantic main.cpp -o Lab9
parsifal@DESKTOP-3G70RV4:~/DA/Lab9\$ cat test0.txt 
5 6 1 5
1 2 2
1 3 0
3 2 10
4 2 1
3 4 4
4 5 5
parsifal@DESKTOP-3G70RV4:~/DA/Lab9\$ ./Lab9 < test0.txt 
8
parsifal@DESKTOP-3G70RV4:~/DA/Lab9\$ cat test.txt 
5 5 1 5
1 2 2
1 3 0
3 2 10
4 2 1
3 4 4
parsifal@DESKTOP-3G70RV4:~/DA/Lab9\$ ./Lab9 < test.txt 
No solution
\end{alltt}
\pagebreak
