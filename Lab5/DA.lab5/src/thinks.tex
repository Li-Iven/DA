\section{Выводы}

Выполнив шестую лабораторную работу по курсу \enquote{Дискретный анализ}, я реализовала свой класс для работы с большими числами (длинной арифметикой), что было полезным опытом, так как в жизни часто возникает необходимость работать с числами, намного превосходящими стандартные int и long long.

Основная суть длинной арифметики состоит в том, что мы представляем число в виде массива, хранящего разряды числа в выбранной нами системе счисления — в данном случае я выбрала систему счисления $10^4$ (набольшую степень десяти, квадрат которой не превышает ограничения типа int).

Данный класс поддерживает все те же операции, что и встроенные типы, но работает с ними намного дольше.

Также я познакомилась с представлением длиннной арифметики в библиотеке GNU MP (написанной на языке С) и сравнила её со своим классом. Так как она использует более оптимизированные алгоритмы, то она оказалась намного быстрее моей программы.
\pagebreak