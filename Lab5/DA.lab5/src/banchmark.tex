\section{Тест производительности}

Операции сравнения выполняются следующим образом: сначала сравниваем размеры массиивов digits — в зависимомти от того, какое число оказалось больше по длине, выводим ответ. Если же они равны, то начинаем пошагово сравнивать каждую цифру массивов, пока не найдём несовпадение. В итоге сложность сравнения — $O(n)$, где n — длина массива digits.

\vspace{\baselineskip}
Для выполнения операций сложения и вычитания (в столбик) мы должны пройтись по всем разрядам двух чисел, учитывая возможность переполнения (на один разряд). Поэтому сложность сложения — $O(max(n,m))$, где n и m — длины двух слагаемых, а вычитания — $O(n)$, где n — длина первого числа (в начале мы сравниваем два числа, и, если первое число окажется меньше второго, то программа вернёт ошибку).

\vspace{\baselineskip}
Умножение также выполняем в столбик. Его сложность выёдет $O(n*m)$, так как мы перемножает каждую цифру первого числа с каждой цифрой второго.

\vspace{\baselineskip}
Деление будем выолнять уголком. Будем по одному разряду брать из делимого, начиная со старшего, и бинарным поиском подбирать разряды частного. Сложность выйдет $O(n*(n + log(BASE)*m))$, где n и m — длины первого и второго чиел, а BASE — основание системы счисления (в нашем случае $10^4$).

\vspace{\baselineskip}
Как уже говрилось в начале, возведение в степень будем выполнять с помощью бинарного возведения в степень со сложностью $O(log(deg)*n^2)$, где deg — показатель степени, а n — длина числа.

\vspace{\baselineskip}
Сраним работу нашего класса больших чисел с библиотекой GNU MP. ("GMP — это свободная библиотека для производства различных арифметических действий над целыми, рациональными и действительными числами. Разрядность чисел, с которыми работает библиотека ограничивется памятью самой машины"\cite{GNU MP}).

{\setlength{\extrarowheight}{7pt}
\begin{center}
\begin{tabular}{|m{4cm}|m{4cm}|m{4cm}|m{4cm}|}
\hline
\rowcolor{pink}
Количество строк в тесте & Длина чисел в тесте & Мой класс, мкс & GNU MP, мкс\\
\hline
\rowcolor{lightgray}
\multicolumn{4}{|c|}{Тесты сложения и вычитания} \\
\hline
$10^3$ & $10^5$ & 23.2 & 15.7 \\
\hline
$10^5$ & $10^2$ & 2469.7 & 919.7 \\
\hline
\rowcolor{lightgray}
\multicolumn{4}{|c|}{Тесты умножения} \\
\hline
$10^3$ & $10^5$ & 93.1 & 21.6 \\
\hline
$10^5$ & $10^2$ & 4915 & 927.9 \\
\hline
$10^5$ & $10^2$ & 2469.7 & 919.7 \\
\hline
\rowcolor{lightgray}
\multicolumn{4}{|c|}{Тесты деления} \\
\hline
$10^3$ & $10^5$ & 2940.8 & 56 \\
\hline
$10^5$ & $10^2$ & 127391 & 2626.3 \\
\hline
\end{tabular}
\end{center}
}

\vspace{\baselineskip}
Как видно, библиотека CNU MP везде выигрывает по времени — так как она использует более оптимизированные алгоритмы: например при делении она намного быстрее находит частное, а при умножении использует алгоритм Карацубы, имеющий сложность $O(n*\sqrt{n})$.

\pagebreak