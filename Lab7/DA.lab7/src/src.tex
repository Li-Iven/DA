\section{Описание}
Как сказано в \cite{wiki}: "оптимальная подструктура в динамическом программировании означает, что оптимальное решение подзадач меньшего размера может быть использовано для решения исходной задачи".

\vspace{\baselineskip}

"В общем случае мы можем решить задачу, в которой присутствует оптимальная подструктура, проделывая следующие три шага:

\begin{enumerate}
    \item Разбиение задачи на подхадачи меньшего размера.
    \item Нахождение оптимального решения подзадач рекурсивно, проделывая такой же трёхшаговый алгоритм.
    \item Использование полученного решения подзадач для конструирования решения исходной задачи.
\end{enumerate}

\vspace{\baselineskip}

Подзадачи решаются делением их на подзадачи ещё меньшего размера и т.д., пока не приходят к тривиальному случаю задачи, решаемой за константное время (ответ можно сказать сразу)." \cite{wiki}

\vspace{\baselineskip}
В данной задаче мы будем делать проход снизу вверх, последовательно считая минимальный штраф для попадания в текущую клетку массива для всех клеток на основе уже посчитанных таким образом минимальных штрафов для клеток внизу. Очевидно, что для клеток со строки n-1 ничего считать не нужно.  

После этого их первой строки выбирается клетка с наименьшим штрафом, а затем алгорит проходит вниз по теблице, выбирая наименьшее число из трёх клеток снизу.

\pagebreak

\section{Исходный код}

\vspace{\baselineskip}

\begin{lstlisting}[language=C++]
#include <iostream>
#include <vector>
#include <algorithm>

using namespace std;

int main() {

    int n, m;
    cin >> n >> m;
    vector<vector<long long>> A (n, vector<long long>(m));

    for(int i = 0; i < n; ++i) {
        for (int j = 0; j < m; ++j) {
            cin >> A[i][j];
        }
    }

    for(int i = n - 2; i > (-1); --i) {
        for (int j = 0; j < m; ++j) {
            if (j == 0) { 
                A[i][j] += min(A[i + 1][j], A[i + 1][j + 1]);
            }
            else if (j == m - 1) {
                A[i][j] += min(A[i + 1][j - 1], A[i + 1][j]);
            }
            else {
                A[i][j] += min({A[i + 1][j - 1], A[i + 1][j], A[i + 1][j + 1]});
            }
        }
    }

    long long minimum = A[0][0];
    int ind = 0;
    for(int i = 1; i < m; ++i) {
        if (A[0][i] <= minimum) {
            minimum = A[0][i];
            ind = i;
        }
    }

    cout << minimum << "\n";
    
    int i, j = ind;
    for(i = 0; i < n - 1; ++i) {
        cout << "(" << i + 1 << "," << j + 1 << ") ";
        if (j == 0) { 
            long long x = min(A[i + 1][j], A[i + 1][j + 1]);
            if (x == A[i + 1][j + 1]) {
                j++;
            }
        }
        else if (j == m - 1) {
            long long x = min(A[i + 1][j - 1], A[i + 1][j]);
            if (x == A[i + 1][j - 1]) {
                j--;
            }
        }
        else { 
            long long x = min({A[i + 1][j - 1], A[i + 1][j], A[i + 1][j + 1]});
            if (x == A[i + 1][j - 1]) {
                j--;
            }
            else if (x == A[i + 1][j + 1]) {
                j++;
            }
        }  
    }
    cout << "(" << i + 1 << "," << j + 1 << ")\n";

    return 0;
}
\end{lstlisting}

\vspace{\baselineskip}

\pagebreak

\section{Консоль}
\begin{alltt}
parsifal@DESKTOP-3G70RV4:~/DA/Lab7\$ g++ main.cpp -o Lab7
parsifal@DESKTOP-3G70RV4:~/DA/Lab7\$ ./Lab7
3 3
3 1 2
7 4 5
8 6 3
8
(1,2) (2,2) (3,3)
parsifal@DESKTOP-3G70RV4:~/DA/Lab7\$ ./Lab7
3 2
-1 2
3 4
5 6
7
(1,1) (2,1) (3,1)
\end{alltt}
\pagebreak
