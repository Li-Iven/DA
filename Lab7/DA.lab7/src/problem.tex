\CWHeader{Лабораторная работа \textnumero 7}

\CWProblem{ Обход матрицы 

\vspace{\baselineskip}

{\bfseries Вариант \textnumero 5}

При помощи метода динамического программирования разработать алгоритм решения задачи, определяемой своим вариантом; оценить время выполнения алгоритма и объем затрачиваемой оперативной памяти. Перед выполнением задания необходимо обосновать применимость метода динамического программирования.

Разработать программу на языке C или C++, реализующую построенный алгоритм. Формат входных и выходных данных описан в варианте задания:

Задана матрица натуральных чисел {\bfseries A} размерности {\bfseries n × m}. Из текущей клетки можно перейти в любую из 3-х соседних, стоящих в строке с номером на единицу больше, при этом за каждый проход через клетку {\bfseries (i, j)} взымается штраф {\bfseries $A_{i,j}$}. Необходимо пройти из какой-нибудь клетки верхней строки до любой клетки нижней, набрав при проходе по клеткам минимальный штраф.

\vspace{\baselineskip}

{\bfseries Формат входных данных} 

Первая строка входного файла содержит в себе пару чисел {\bfseries 2 $\geq$ n $\geq$ 1000} и {\bfseries 2 $\geq$ m $\geq$ 1000}, затем следует {\bfseries n} строк из {\bfseries m} целых чисел.

\vspace{\baselineskip}

{\bfseries Формат результата} 

Необходимо вывести в выходной файл на первой строке минимальный штраф, а на второй — последовательность координат из {\bfseries  n} ячеек, через которые пролегает маршрут с минимальным штрафом.

}

\pagebreak